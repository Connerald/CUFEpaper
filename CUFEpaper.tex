%% CUFEpaper.tex
%% Copyright 2025 Connerald
%
% This work may be distributed and/or modified under the
% conditions of the LaTeX Project Public License, either version 1.3
% of this license or (at your option) any later version.
% The latest version of this license is in
%   https://www.latex-project.org/lppl.txt
% and version 1.3c or later is part of all distributions of LaTeX
% version 2008 or later.
%
% This work has the LPPL maintenance status `maintained'.
% 
% The Current Maintainer of this work is Connerald.
%
% This work consists of the files CUFEpaper.tex and CUFEpaper.cls
% and the derived file CUFEpaper.pdf.

% 本模板遵循 LaTeX Project Public License (LPPL),允许自由使用、分发和修改,但请保留本版权声明。
% 本模板按“原样”提供,不附带任何明示或暗示的担保,包括但不限于适销性或特定用途适用性的担保。使用本模板所产生的任何后果,作者不承担任何责任。

\documentclass[UTF-8]{CUFEpaper}

\term{2024-2025学年第二学期}
\course{投资学}
\code{ALPHA191}
\teacher{科比·布莱恩特}
\name{牢大}
\stuid{20250509}
\class{雷军班}
\score{}
\grader{}
\papertitle{浅析完善国有企业内部控制制度}

\begin{document}

\makecover % 制作封面

\begin{abstract}
第三方物流是指供方与需方以外的物流企业提供物流服务的业务模式,在我国入世之后,这个新兴领域将面临国际竞争者的严峻挑战。为了促进我国工商企业运营效率的整体提高,我国的第三方物流企业应迅速改善自身技术水平和服务条件以加强竞争力。

本文首先对我国第三方物流的现状进行了分析,认为提供优质服务的第三方物流发展潜力巨大。此外,在国际环境中,第三方物流需求量会随着我国加入WTO大幅增加,但前提必须是我们的第三方物流企业有能力满足跨国公司的需求。

本文针对以上问题提出了解决方案,认为应该完善基础设施,加强信息化建设,并要建立物流联盟,提高企业的规格和信用度,同时注重培养物流人才,并适时发展第四方物流与绿色物流。

\textbf{关键词:} 第三方物流\quad 中国物流业现状\quad 存在问题\quad 解决方法
\end{abstract}

% 英文摘要
\makeenabstract{%
Third-party logistics refers to a business model in which logistics services are provided by enterprises other than the supplier and the demander. After China's accession to the WTO, this emerging field will face severe challenges from international competitors. In order to improve the overall operational efficiency of Chinese industrial and commercial enterprises, domestic third-party logistics companies should quickly improve their technical level and service conditions to enhance competitiveness.
This paper first analyzes the current situation of third-party logistics in China and believes that there is great potential for the development of high-quality third-party logistics services. In addition, in the international environment, the demand for third-party logistics will increase significantly as China joins the WTO, but the premise is that our third-party logistics companies must be able to meet the needs of multinational companies.
In response to the above problems, this paper proposes solutions such as improving infrastructure, strengthening informatization, establishing logistics alliances, improving enterprise standards and credibility, focusing on cultivating logistics talents, and developing fourth-party logistics and green logistics in due course.
}{third-party logistics\quad China logistics industry\quad existing problems solutions}

\tableofcontents
\newpage

% 正文首页居中标题
\makeatletter
\begin{center}
  {\zihao{-3}\bfseries \@papertitle}
\end{center}
\makeatother

德国经济学家曾提出,未来只有三种人:生产者,消费者和流通者,可见物流业已经成为影响一切工商业发展的重要领域。而第三方物流正是物流业发展到高级阶段的产物,它有利于企业集中主业,减少库存和节省费用。但由于我国的第三方物流只处于成长阶段,在观念、经营规模、成本控制、服务质量、从业人员素质等方面与国际同业还有着较大的差距,因此,如何根据中国的国情迅速健全我们的第三方物流业就成为了入世后急需解决的问题。本文以广义的第三方物流为阐述对象,希望通过对其现状及问题的分析找到突破口,增强TPL的竞争能力,进而提升工商企业整体的经济效益。

\section{我国第三方物流的概况}
我国《物流术语》定义的第三方物流(Third-Party Logistics,简称3PL或TPL)是指供方与需方以外的物流企业提供物流服务的业务模式。指在物流渠道中由中间商以合同形式在一定期限内向供需企业提供所需要的全部或部分物流服务。第三方物流企业在货物的实际供应链中并不是一个独立的参与者,而是代表发货人或收货人,通过提供一整套物流活动来服务于供应链 。而我国目前的物流公司大多只提供分割的、部分环节上的流通运输服务,远没有发挥真正的TPL的功效和优势。
\subsection{市场需求远期潜力很大,但近期容量较小}
1.第三方物流的有效需求不足

自营物流的比例很大,有待我们的物流企业去主动开发和挖掘潜在的客户需求。2003年中国仓储协会主持进行了第四次中国物流市场调研活动,从中可以看出,在工商企业物流执行的主体方面:

(1)生产企业原材料物流执行主题主要是供货方,占53\%;第三方仅占22\%。①第三方所占比例呈上升趋势。②成品销售物流中,39\%的执行主体是公司,仅有17\%全部是第三方,44\%的公司同时采用两种形式。

(2)商业企业物流执行主体的47\%有第三方参与,11\%的企业只由供货方提供服务,19\%的企业由公司自理,与上次调查相比,第三方物流的比例有较大上升。

2.第三方物流的要求越来越高
\subsection{物流市场需求的结构性不足}
\subsubsection{这是三级标题}
正文\ ......


\section{相关工作}
近年来,国内外学者对投资组合理论、资产定价模型以及行为金融等领域进行了大量研究。Markowitz 提出的均值-方差模型为现代投资组合理论奠定了基础,Fama 和 French 等学者则进一步完善了资产定价理论。此外,行为金融学的兴起也为解释市场异常现象提供了新的视角。本文将在前人研究的基础上,结合我国市场的实际情况进行分析。

\section{方法}
本文主要采用文献分析法与实证研究法。首先,通过梳理相关文献,明确投资学领域的研究现状与发展趋势。其次,选取沪深股市的部分样本数据,运用均值-方差模型和CAPM模型进行实证分析。数据处理与分析主要借助Excel和Python等工具完成。

\section{实验}
在实验部分,本文选取了2020年至2024年沪深300指数成分股的月度收益率数据,计算各资产的期望收益与风险,并构建最优投资组合。同时,利用CAPM模型对样本股票进行回归分析,检验市场风险溢价的显著性。实验过程中,所有数据均来自Wind数据库,确保数据的权威性与准确性。

\section{结果与分析}
实证结果显示,均值-方差模型能够有效指导投资组合的构建,分散风险的同时提升收益水平。CAPM模型回归结果表明,大部分样本股票的β系数显著为正,市场风险溢价存在,但部分股票存在异常收益,说明市场并非完全有效。结合行为金融学理论,可以解释部分投资者的非理性行为对市场价格的影响。

\section{结论}
本文通过理论与实证相结合的方法,分析了投资组合优化与资产定价模型在我国市场的适用性。研究发现,传统投资理论在实际应用中具有一定指导意义,但也需结合市场实际和投资者行为进行调整。未来研究可进一步引入更多行为金融因素,提升模型的解释力和实用性。

\end{document}

